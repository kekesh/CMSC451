\newpage
\section{Thursday, February 6, 2020}

Today, we'll discuss algorithms to find articulation points and biconnected components. 

\subsection{Articulation Points}

\begin{definition}
An \vocab{articulation point} or \vocab{cut vertex} is a vertex in a graph $G = (V, E)$ whose removal (along with any incident edges) would disconnect $G$. 
\end{definition}

\begin{definition}
A graph is said to be \vocab{biconnected} if the graph not have any articulation points. 
\end{definition}

For example, consider the following graph:

\begin{figure}[h]
\centering
\begin{tikzpicture}
\begin{scope}[every node/.style={circle,thick,draw}]
    \node (A) at (0,0) {A};
    \node (B) at (0,3) {B};
    \node (C) at (2.5,4) {C};
    \node (D) at (2.5,1) {D};
    \node (E) at (2.5,-3) {E};
    \node (F) at (5,3) {F} ;
\end{scope}

\begin{scope}[>={Stealth[black]},
              every node/.style={fill=white,circle},
              every edge/.style={draw=red,very thick}]
    \path [-] (A) edge (B);
    \path [-] (B) edge (C);
    \path [-] (A) edge (D);
    \path [-] (D) edge (C);
    % \path [-] (A) edge (E);
    \path [-] (D) edge (E);
    \path [-] (D) edge (F);
    \path [-] (C) edge (F);
    % \path [-] (E) edge (F); 
    % \path [-] (B) edge[bend right=40] (F); 
\end{scope}
\end{tikzpicture}
\caption{A Graph with an Articulation Point}
\end{figure}

In the diagram above, Vertex $D$ is an articulation point. To see why, note that if we were to remove Vertex $D$ (and any incident edges to $D$) from the graph, then we would end up with two connected components: the first component would contain the vertices $A, B, C,$ and $F$, whereas the second component would only contain the vertex $E$. \\

Why are articulation points important? One example in which searching for articulation points is important is in the study of networks. In a network modeled by a graph, an articulation point represents a vulnerability: it is a single point whose failure would split the network into two or more components (preventing communication between the nodes in different networks). 
 \\
 
 
 How do we find an articulation points? The brute force algorithm is as follows:
 
 \begin{enumerate}
     \item Run an $\mathcal{O}(V + E)$ depth-first search or breadth-first search to count the number of connected components in the original graph $G = (V, E)$. 
     \item For each vertex $v\in V$, remove $v$ from $G$, and remove any of $v$'s incident edges. Run an $\mathcal{O}(V + E)$ depth-first search or breadth-first search again, and check if the number of connected components increases. If so, then $v$ is an articulation point. Restore $v$ and any of its incident edges.
 \end{enumerate}
 
This naive algorithm calls the depth-first search or breadth-first search algorithm $\mathcal{O}(V)$ times. Hence, it runs in $\mathcal{O}(V \times (V + E)) = \mathcal{O}(V^2 + VE)$ time. \\
 
 
While this algorithm \textit{works}, it is not as efficient as we can get. We will now describe a linear-time algorithm that runs the depth-first search algorithm just \textit{once} to identify all articulation points and bridges. \\
 
 
In this modified depth-first search, we will now maintain two numbers for each vertex $v$: \verb!dfs_num(v)! and \verb!dfs_low(v)!. The quantity \verb!dfs_num(v)! represents a label that we will assign to nodes in an increasing fashion. For instance, the vertex from which we call depth-first search would have a \verb!dfs_num! of $0$. The subsequent vertex we visit would be assigned a \verb!dfs_num! of $1$, and so on. \\
 
On the other hand, the quantity \verb!dfs_low(v)!, also known as the \vocab{low-link value} of the vertex $v$, represents the smallest \verb!dfs_num! reachable from that node while performing a depth-first (including itself). \\
 
Here's an example. Consider the following directed graph: 
  
  
\begin{figure}[h]
\centering
\begin{tikzpicture}
\begin{scope}[every node/.style={circle,thick,draw}]
    \node (A) at (0,0) {A};
    \node (B) at (2,0) {B};
    \node (C) at (4,0) {C};
    \node (D) at (6,0) {D};
    % \node (E) at (2.5,-3) {E};
    % \node (F) at (5,3) {F} ;
\end{scope}

\begin{scope}[>={Stealth[black]},
              every node/.style={fill=white,circle},
              every edge/.style={draw=red,very thick}]
    \path [-] (A) edge (B);
    \path [-] (B) edge (C);
    \path [-] (C) edge (D);
    \path [-] (C) edge[bend right=40] (A); 

    % \path [-] (A) edge (D);
    % \path [-] (D) edge (C);
    % \path [-] (A) edge (E);
    % \path [-] (D) edge (E);
    % \path [-] (D) edge (F);
    % \path [-] (C) edge (F);
    % \path [-] (E) edge (F); 
    % \path [-] (B) edge[bend right=40] (F); 
\end{scope}
\label{artpoint:ex}
\end{tikzpicture}
\caption{Articulation Point Example}
\end{figure}

Suppose we perform a depth-first search starting at Vertex $A$. 

\begin{itemize}
    \item Vertex $A$ will be assigned a \verb!dfs_num! of $0$ since this is the first vertex that we're visiting. Moreover, $0$ is the smallest \verb!dfs_num! that is reachable from $A$ (all other vertices have their \verb!dfs_num! set to \verb!nil! or \verb!INFINITY!). Hence, we set \verb!dfs_num(A) = 0! and \verb!dfs_low(A) = 0!. 
    \item Next, we visit vertex $B$. Vertex $B$ is assigned a \verb!dfs_num! of $1$ since it's the second vertex we're visiting. Moreover, Vertex $B$ has a \verb!dfs_low! value of $0$ since we can reach a vertex with a \verb!dfs_num! value of $0$ through the path $B \rightarrow C \rightarrow A$. Note that it would be invalid to say that the path $B \rightarrow A$ causes \verb!dfs_low(B)! to equal $0$ since we cannot go backwards in the depth-first search traversal. 
    \item Applying similar reasoning, we find that vertex $C$ ends up with a \verb!dfs_num! value of $2$, and it also has a \verb!dfs_low! value of $0$ (we can reach vertex $A$).
    \item Finally, vertex $B$ ends up with a \verb!dfs_num! value of $3$; however, no vertices with a lower \verb!dfs_num! value are reachable from $D$. Hence, the \verb!dfs_low! value of $D$ is also equal to $3$. Note that it is incorrect to say that $D$ has a \verb!dfs_low! value of $0$ through the path $D\rightarrow C\rightarrow A$ since we cannot revisit vertices while performing the depth-first search algorithm. 
\end{itemize}


Why do we care about these \verb!dfs_num! and \verb!dfs_low! values? It becomes more clear when we consider the depth-first search tree produced by calling the depth-first search algorithm. The quantity \verb!dfs_low(v)! represents the smallest \verb!dfs_num! value reachable from the current depth-first search spanning subtree rooted at the vertex $v$.  The value \verb!dfs_low(v)! can only be made smaller if there's a back edge (an edge from a vertex $v$ to an ancestor of $v$) in the depth-first search tree. \\

This leads us to make the following observation: If there's a vertex $u$ with neighbor $v$ satisfying \verb!dfs_low(v) >= dfs_num(u)!, then we can conclude that vertex $u$ is an articulation point.  Note that this makes sense intuitively since it means that the \textit{smallest} numbered vertex that we can ever reach starting from vertex $v$ is greater than or equal to the number we assigned to $u$. Hence, removing $u$ would disconnect \verb!v! from any vertex with smaller \verb!dfs_num! than \verb!dfs_num(u)!. \\


Going back to the previous graph figure, we can note that the following:
\[
\verb!3 = dfs_num(D) >= dfs_low(C) = 0!
\]
As stated previously, this implies that Vertex $C$ is an articulation point. Note that removing Vertex $C$ would disconnect the vertices $A$ and $B$ from Vertex $D$.  \\

Now, there's one special case to this algorithm. The root of the depth-first search spanning tree (the vertex that we choose as the source in the first depth-first search call) is an articulation point only if it has more than one children. This one case is not detected by the algorithm; however, it is easy to check in implementation. \\

% When actually implementing the algorithm, we can assign \verb!dfs_num! values ``on-the-fly" when we visit the vertex. A C++ implementation is provided below: \\

% \begin{lstlisting}
% int dfsCounter; /* Used to assign vertices their dfs_num value. */
% int dfs_num[10000]; /* Stores the dfs_num values of each vertex. */
% int dfs_lo[10000]; /* Stores the dfs_low values of each vertex. */
% int color[10000]; /* Check if we've visited the vertex yet. */
% int parent[10000]; /* We maintain the parent of a vertex so we don't update the dfs_low value with an already-visited vertex. */
% bool isArticulationPoint[10000];


% /* O(V + E) Articulation Point Algorithm. */
% void dfs(int u, const vector<vector<int>>& AdjList) {
%   color[u] = 1;
%   dfs_num[u] = dfs_lo[u] = dfsCounter++;
  
%   for (int i = 0; i < AdjList[u].size(); i++) {
%     int v = AdjList[u][i];
%     if (!visited[v]) {
%       /* Vertex v hasn't been visited. */
%       parent[v] = u;
%       dfs(v, AdjList);

%       if (dfs_lo[v] >= dfs_num[u]) {
%         // Vertex u is an articulation point. 
%         isArticulationPoint[u] = true;
%       }
%       /* Update the dfs_lo value of u. */
%       dfs_lo[u] = MIN(dfs_lo[u], dfs_lo[v]);
%     } else if (color[v] == 1 && parent[u] != v) {
%       /* (u, v) is a back edge in the depth-first search tree. */
%       /* We've already visited vertex v, so we just update the dfs_lo value. */
%       dfs_lo[u] = min(dfs_lo[u], dfs_num[v]);
%     }
%   }
%   color[u] = 2;
% }
% \end{lstlisting}

