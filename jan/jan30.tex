\newpage
\section{Thursday, January 30, 2020}

\subsection{Optimality and Correctness of Gale-Shapley}


Last time, we introduced the Gale-Shapley algorithm to find a stable matching. Today, we'll prove that the algorithm is correct (i.e. it never produces an unstable matching), and it is optimal for men (i.e. the men always end up for their preferred choice). \\

First, we'll show that the algorithm is correct: \\

\begin{proposition}
The matching generated by the Gale-Shapley algorithm is never an unstable matching.  
\end{proposition}

\begin{proof}
Suppose, for the sake of contradiction, that $m$ and $w$ prefer each other over their current partner in the matching generated by the Gale-Shapley algorithm. This can happen either if $m$ never proposed to $w$, or if $m$ proposed to $w$ and $w$ rejected $m$. In the former case, $m$ must prefer his partner to $w$, which implies that $m$ and $w$ do not form an unstable pair. In the latter case, $w$ prefers her partner to $m$, which also implies $m$ and $w$ don't form an unstable pair. Thus, we arrive at a contradiction. 
\end{proof}

Next, we'll prove that the algorithm is optimal for men. However, before presenting the proof, observe that it is not too hard to see intuitively that the algorithm ``favors" the men. Since the men are doing all of the proposing and the women can only do the deciding, it turns out that the men always ends up with their most preferred choice (as long as the matching remains stable).

\begin{proposition}
The matching generated by the Gale-Shapley algorithm gives men their most preferred woman possible without contradicting stability. 
\end{proposition}

\begin{proof}
 \\



To see why this is true, let $A$ be the matching generated by the men-proposing algorithm, and suppose there exists some other matching $B$ that is better for at least one man, say $m_0$. If $m_0$ is matched in $B$ to $w_1$ which he prefers to his match in $A$, then in $A$, $m_0$ must have proposed to $w_1$ and $w_1$ must have rejected him. This can only happen if $w_1$ rejected him in favor of some other man --- call him $m_2$. This means that in $B$, $w_1$ is matched to $m_0$ but she prefers $m_2$ to $m_0$. Since $B$ is stable, $m_2$ must be matched to some woman that he prefers to $w_1$; say $w_3$. This means that in $A$, $m_2$ proposed to $w_3$ before proposing to $w_1$, and this means that $w_3$ rejected him. Since we can perform similar considerations, we end up tracing a ``cycle of rejections" due to the finiteness of the sets $A$ and $B$. 
\end{proof}

